\documentclass{article}
\usepackage{graphicx}
\usepackage{setup/jfrExamplee}
\usepackage{setspace}
\usepackage{setup/apalike}


% --- Tikz ---
\usepackage{tikz}
\usetikzlibrary{shapes,arrows}
\usepackage{verbatim}
\usepackage{tikz,pgfplots}
\usetikzlibrary{positioning,shapes.misc} % for uneven round corner rectangles

\usetikzlibrary{arrows.meta} % to draw customized arrow size

\newlength\figureheight 
\newlength\figurewidth

\usetikzlibrary{plotmarks} 




% --- fonts ---

\usepackage{amsmath}
%\usepackage[math]{kurier}
% \usepackage{enumitem,showframe}
\usepackage{amssymb}
\usepackage{soul}
\usepackage{nicefrac} % for inline fraction
\usepackage{stmaryrd} % maps from


% --- figure ---
\usepackage{dblfloatfix}
\usepackage{fixltx2e} % two column figure
\usepackage{subfigure} % subfigure

\usepackage{stackengine}

%\usepackage{subcaption}
\usepackage{blindtext}

\usepackage[export]{adjustbox} %for box on figure


\DeclareMathAlphabet{\mathpzc}{OT1}{pzc}{m}{it}



% --- Algorithm
%\usepackage{algpseudocode}
%\usepackage{algorithmicx}
%\usepackage{algorithm}

%\usepackage[]{algorithm2e}

\usepackage{algorithm}
\usepackage{algorithmic}
\usepackage{boxedminipage}

\usepackage{amsthm}
\newtheorem{lemma}{Lemma}
\newtheorem{definition}{Definition}
\usepackage{longtable}

\def\Var{\mathop{\rm Var}} % define variance symbol
\def\pSPP{$\psi c$-SPP}



% --- table ---
\usepackage{multirow} % multi rows

%\usepackage{array}
%\newcolumntype{L}[1]{>{\raggedright\let\newline\\\arraybackslash\hspace{0pt}}m{#1}}
%\newcolumntype{C}[1]{>{\centering\let\newline\\\arraybackslash\hspace{0pt}}m{#1}}
%\newcolumntype{R}[1]{>{\raggedleft\let\newline\\\arraybackslash\hspace{0pt}}m{#1}}
% ---

\usepackage{caption}
%\usepackage[subrefformat=parens,labelformat=parens]{subcaption} %% provides subfigure
\usepackage{tabularx} %% tabularx and multirow so that we can do nice subfigure layouts
%\usepackage[margin=1in]{geometry}


\usepackage{array}
\newcolumntype{L}[1]{>{\raggedright\let\newline\\\arraybackslash\hspace{0pt}}m{#1}}
\newcolumntype{C}[1]{>{\centering\let\newline\\\arraybackslash\hspace{0pt}}m{#1}}
\newcolumntype{R}[1]{>{\raggedleft\let\newline\\\arraybackslash\hspace{0pt}}m{#1}}
% ---



% --- list ---
%http://texblog.org/2013/02/01/inline-lists-in-latex-using-paralist/
%\documentclass[11pt]{article}
\usepackage{paralist}
%\begin{document}
%This includes:
%\begin{inparaenum}[1)]
%\item first task,
%\item second task and
%\item third task
%\end{inparaenum}
%to be completed by the end of the month.
%\end{document}

%\documentclass[11pt]{article}
%\usepackage{paralist}
%\begin{document}
%This includes:
%\begin{inparaitem}
%\item first task,
%\item second task and
%\item third task
%\end{inparaitem}
%to be completed by the end of the month.
%\end{document}

%\documentclass[11pt]{article}
%\usepackage{paralist}
%\begin{document}
%This includes:
%\begin{inparadesc}
%\item[Task] first task,
%\item[Task] second task and
%\item[Task] third task
%\end{inparadesc}
%to be completed by the end of the month.
%\end{document}
% ---------------------------------------- %


% --------------- frame box for text ------------------------------
% 
\newcommand\FramedBox[3]{%
	\setlength\fboxsep{0pt}
	\fbox{\parbox[t][#1][c]{#2}{\centering\scriptsize #3}}}


\makeatletter
\newcommand{\mybox}{%
	\collectbox{%
		\setlength{\fboxsep}{1pt}%
		\fbox{\BOXCONTENT}%
	}%
}
\makeatother  

\newcommand{\cfbox}[2]{%
	\colorlet{currentcolor}{.}%
	{\color{#1}%
		\fbox{\color{currentcolor}#2}}%
}
%----------------------------------------------------------------------


% --- shorts
\usepackage{xspace} % space after \newcommand
%\newcommand{\dg}{$^\circ$\xspace} % degree
\newcommand{\cSPP}{$conv$-SPP\xspace} % conv-SPP
%\newcommand{\sSPP}{$\psi c$-SPP\xspace}
\def\Var{\mathop{\rm Var}} % variance
\def\xSPP{{\normalfont{$\mathsf{x} v t$-SPP}}\xspace} % xvt-SPP


\newcommand{\dg}{$^\circ$\xspace} % degree
\newcommand{\ddg}{^\circ} % degree
%\newcommand{\cSPP}{$conv$-SPP\xspace} % conv-SPP
%\newcommand{\sSPP}{$\psi c$-SPP\xspace}


\DeclareMathOperator*{\argmin}{\arg\!\min} %argmin
\DeclareMathOperator*{\argmax}{\arg\!\max} %argmax


\definecolor{color-figbox}{gray}{1.0} % figure box color
\def \figWidthSample {7.0cm} % figure width % 7.0cm
\def \figWidthTestRecon {3.5cm} % figure width
\def \figWidthCorridor {10.0cm} % figure width
\def \figWidthForest {7.0cm} % figure width
\def \figWidthCastings {12.0cm} % figure width




% --- changes
\usepackage[color=green]{changes} % highlight the changes


%% Uncomment line below for double spacing
%\doublespacing

%this template built off template for NIPS 2004
